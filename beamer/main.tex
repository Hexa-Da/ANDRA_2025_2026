\documentclass[aspectratio=169]{beamer}

% --- PAQUETS ---
\usepackage[utf8]{inputenc}
\usepackage[T1]{fontenc}
\usepackage[french]{babel}
\usepackage{graphicx}
\usepackage{graphbox}
\usepackage{booktabs}
\usepackage{tikz}
\usepackage{caption}

% --- THÈME ---
\usetheme{Madrid} 
\usecolortheme{whale} 

% --- CONFIGURATION DES IMAGES ---
\graphicspath{{images/}} 

% --- INFORMATIONS ---
\title[Robotique \& IA - ANDRA]{Robotique et IA en environnements complexes}
\subtitle{Exploration de la galerie souterraine de l'ANDRA}
\author[Guichard, Lefebvre, Dame, Paris]{Vincent GUICHARD, Adrien LEFEBVRE\\Lucas DAME, Paul-Antoine PARIS}
\institute[Mines Nancy]{École des Mines de Nancy \\ Partenaire : ANDRA}
\date{Mi-Soutenance - Janvier 2026}

% --- LOGOS PAGE DE TITRE ---
\titlegraphic{
    \vspace{-0.5cm}
    \includegraphics[align=c, height=1.5cm]{images/logo_andra.png}%
    \includegraphics[align=c, height=3cm]{images/logo_mines_nancy.png}%
}

\begin{document}

% --- SLIDE 1 : TITRE ---
\begin{frame}
    \titlepage
\end{frame}

% --- SLIDE 2 : SOMMAIRE ---
\begin{frame}{Sommaire}
    \tableofcontents
\end{frame}

% --- SECTION 1 ---
\section{Contexte et Enjeux}

% --- SLIDE 3 : CONTEXTE ---
\begin{frame}{Contexte du Projet}
    \begin{columns}
        \column{0.5\textwidth}
        \textbf{Le Projet CIGEO :}
        \begin{itemize}
            \item Stockage géologique à \textbf{500m de profondeur} (Bure).
            \item Infrastructure : Plus de \textbf{270 km de galeries}.
        \end{itemize}
        
        \vspace{0.5cm}
        
        \textbf{La Problématique :}
        \begin{itemize}
            \item Apparition de fissures (cisaillement géologique).
            \item Inspection manuelle répétitive et chronophage.
        \end{itemize}
        
        \column{0.5\textwidth}
        \begin{figure}
            \centering
            \includegraphics[width=0.9\textwidth]{cigeo.jpg}
            \caption{Vue du laboratoire à Bure}
        \end{figure}
    \end{columns}
\end{frame}

% --- SECTION 2 ---
\section{État de l'Art et Architecture}

% --- SLIDE 4 : DÉTECTION ---
\begin{frame}{État de l'Art : Détection par IA}
    \begin{columns}[t]
        \column{0.6\textwidth}
        \begin{block}{Limites historiques (avant 2024)}
            Les modèles "universels" confondaient fissures, câbles et tuyaux (taux élevé de faux positifs).
        \end{block}

        \vspace{0.3cm}

        \textbf{Avancées 2024-2025 :}
        \begin{itemize}
            \item Création d'un \textbf{dataset spécifique} ($\approx$ 300 images).
            \item Utilisation du modèle \textbf{YOLOv11}.
            \item \textbf{Résultat :} Distinction efficace fissures.
        \end{itemize}

        \column{0.27\textwidth}
        \vspace*{-0.7cm}
        \begin{figure}
            \centering
            \includegraphics[width=0.9\textwidth]{example_mauvaise_detection_fissure_2.png}
            \caption{Exemple confusion}
        \end{figure}
    \end{columns}
\end{frame}

% --- SLIDE 5 : NAVIGATION ---
\begin{frame}{État de l'Art : Navigation \& Positionnement}
    \textbf{Architecture Logicielle :}
    \begin{itemize}
        \item Migration complète vers \textbf{ROS2 (Humble)}.
        \item Abandon des API propriétaires $\rightarrow$ Modularité accrue.
    \end{itemize}

    \vspace{0.5cm}

    \textbf{Stratégie de Localisation :}
    \begin{itemize}
        \item \textbf{Problème :} Le SLAM (cartographie simultanée) dérive trop en tunnel ("Drift").
        \item \textbf{Solution retenue : AMCL} (Adaptive Monte Carlo Localization).
        \item \textit{Principe :} Se recaler sur une carte statique connue via le LiDAR 2D.
    \end{itemize}
\end{frame}

\begin{frame}{Architecture Matérielle}
    \begin{columns}[t] 
        % --- COLONNE 1 : TEXTE ---
        \column{0.30\textwidth}
        \vspace{-0.5cm} 
        
        \textbf{Le Cerveau :}
        \begin{itemize}
            \item Jetson Orin Nano
        \end{itemize}
        
        \textbf{La Base :}
        \begin{itemize}
            \item Agilex Scout Mini
            \item Contrôle CAN
        \end{itemize}

        \textbf{La Caméra :}
        \begin{itemize}
            \item Caméra PTZ
        \end{itemize}
        
        \textbf{Les Capteurs :}
        \begin{itemize}
            \item LiDAR 2D
            \item ZED2 (IMU)
        \end{itemize}

        % --- COLONNE 2 : IMAGE AVANT ---
        \column{0.35\textwidth}
        \vspace{0pt} 
        \begin{figure}
            \centering
            \includegraphics[width=\linewidth, height=5.5cm, keepaspectratio]{images/avant_robot.jpg}
            \caption{Vue Avant}
        \end{figure}

        % --- COLONNE 3 : IMAGE ARRIERE ---
        \column{0.35\textwidth}
        \vspace{0pt}
        \begin{figure}
            \centering
            \includegraphics[width=\linewidth, height=5.5cm, keepaspectratio]{images/arriere_robot.jpg}
            \caption{Vue Arrière}
        \end{figure}
    \end{columns}
\end{frame}

% --- SECTION 3 ---
\section{Travaux Semestre 1}

% --- SLIDE 7 : RECONSTRUCTION ---
\begin{frame}{Reconstruction Système}
    \begin{block}{Incident Critique}
        Perte totale de l'image système de l'année précédente sans sauvegarde.
    \end{block}

    \vspace{0.3cm}

    \textbf{Actions correctives réalisées :}
    \begin{itemize}
        \item Reconstruction intégrale de l'environnement ROS2 (Workspaces, dépendances).
        \item \textbf{Interface CAN :} Patch du driver \texttt{scout\_base} pour reconnaissance matérielle spécifique.
        \item \textbf{Caméra PTZ :} Rétablissement IP et création de scripts d'automatisation.
    \end{itemize}

    \vspace{0.3cm}
    \textbf{État des capteurs :}
    \begin{itemize}
        \item[\textcolor{green}{\checkmark}] Caméra ZED2, PTZ, Odométrie Base.
        \item[\textcolor{red}{\textbf{!}}] \textbf{LiDAR :} Erreurs d'initialisation (Panne matérielle suspectée).
    \end{itemize}
\end{frame}

% --- SLIDE 8 : VISION 360 ---
\begin{frame}{Innovation 2026 : Projet Parallèle 360°}
    \begin{columns}
        \column{0.6\textwidth}
        \textbf{Pourquoi ?}
        \begin{itemize}
            \item La caméra PTZ impose des arrêts fréquents.
            \item Risque de rater des zones hors champ.
        \end{itemize}

        \vspace{0.3cm}

        \textbf{Nouvelle Approche :}
        \begin{itemize}
            \item \textbf{Vision 360° :} Capture globale (Voûte + Murs) en une seule prise.
            \item \textbf{Détection continue :} Analyse en temps réel sans arrêt du robot.
        \end{itemize}

        \column{0.3\textwidth}
        \begin{figure}
            \centering
            \includegraphics[width=1\textwidth]{images/ricoh.jpg}
            \caption{Ricoh Theta}
        \end{figure}
    \end{columns}
\end{frame}

% --- SECTION 4 ---
\section{Perspectives}

% --- SLIDE 9 : OBJECTIFS FEVRIER ---
\begin{frame}{Objectifs : Descente Février 2026}
    \begin{block}{1. Validation Système}
        Tester la fiabilité du robot "reconstruit" en conditions réelles.
        \\ \textit{Cycle : Avancer $\rightarrow$ Stop $\rightarrow$ Photo $\rightarrow$ Repartir.}
    \end{block}

    \vspace{0.5cm}

    \begin{block}{2. Acquisition de Données}
        \begin{itemize}
            \item Focus prioritaire sur la capture \textbf{360°}.
            \item Constitution du dataset pour le nouveau modèle IA.
        \end{itemize}
    \end{block}
\end{frame}

% --- SLIDE 10 : LIMITATIONS ---
\begin{frame}{Contraintes et Limitations Actuelles}
    \begin{itemize}
        \setlength\itemsep{1em}
        \item[\textbf{--}] \textbf{Matériel 360° :} Équipement prêté par l'ANDRA intégration au robot impossible.
        
        \item[\textbf{--}] \textbf{Communication :} Aucune solution validée pour le contrôle sans fil en tunnel profond.
        
        \item[\textbf{--}] \textbf{Maturité :} Système global en retrait par rapport à la fin du projet 2024-2025.

        \item[\textbf{--}] \textbf{Localisation :} La chaîne complète (création de carte + navigation) reste à valider.
    \end{itemize}
\end{frame}

% --- SLIDE 11 : CONCLUSION ---
\begin{frame}{Conclusion}
    \textbf{Bilan Semestre 1 :}
    \begin{itemize}
        \item Socle logiciel ROS2 sain, modulaire et documenté.
        \item Infrastructure de développement rétablie après l'incident système.
    \end{itemize}

    \vspace{0.5cm}

    \textbf{Prochaines étapes (Court terme) :}
    \begin{enumerate}
        \item Réussir l'acquisition de données en février.
        \item Entraîner le modèle sur images 360°.
        \item Atteindre l'état final du projet 2024-2025.
    \end{enumerate}
    
    \vspace{1cm}
    \centering
    \Large \textbf{Merci de votre attention.}
\end{frame}

\end{document}